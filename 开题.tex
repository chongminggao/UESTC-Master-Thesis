现有的推荐论文,主要突破和创新都在推荐方法上,试图用最新的科技,融合能收集到的跨域数据与知识,将数据结构表征和后续推荐结合起来,达到互利的效果。然而大多数方法对数据都有较强的假设,而现实中很多原始数据需要经过处理或转换才能才能完成进一步的挖掘工作。例如,从每天的人类出行、动物迁徙,以及飓风、洋流演化都能采集到大量时空属性的轨迹数据,而轨迹数据天生具有非结构性:其长度不固定,采样率不固定,并且含有很多噪声与不确定性,这使得后续的模式分析和地点、路径推荐等任务变得艰难。为此,在本文中,我将对轨迹数据基本处理,包括轨迹的表征、压缩、以及检索查询等关键步骤做出探讨,并提出对应的高效算法。此外,针对现有的地点推荐系统中面临的问题:数据稀疏、信息偏颇、迭代效率低下,以及不能充分利用多源进行很好的相互促进,我将融合用户的关系网络数据、访问地点的地理层级结构,考虑用户在特定时间、地点的签到信息(check-in),设计出一个全面的推荐系统。通过该推荐系统,用户的隐形偏好能被更好的挖掘,用户不喜欢的数据能够从海量的未访问样本中区分出来。为了克服传统的矩阵分解技术的解释性差、迭代慢的问题,我将采用分步分解的思想。最终,用户与地点的交互关系又将促进用户网络的演化,使得社团挖掘能够取得更好的结果。


近年来,随着移动设备飞快普及与硬件存储、计算能力的飞快提升,每天都有海量的数据以惊人的速度产生。高效地对这些数据进行表征和挖掘,将在现有的经济、环境、生物、医学、交通等领域中产生巨大的效益。具体地,在人们的日常生活中,由数据挖掘带来的效益表现为各式推荐服务。例如基于时间与空间的地点、路径推荐系统,都旨在抓住用户偏好以及领域特性,给用户推荐出合理、准确的结果。然而,涉及到时空的原始轨迹数据通常有着长度不固定、采样率不固定等天生的缺陷,虽然各种研究方法分别基于自己的假设提出了很多独立的轨迹特征提取以及预处理手段,但在富含多源语义的各种应用场景下,这些方法都只考虑了轨迹数据的单反面或少数方面信息,没有将多源信息有机地结合起来。如果能提出一种综合多源语义信息的轨迹表征方式,后续的挖掘任务,例如轨迹压缩、轨迹检索以及相似轨迹推荐都能被极大程度地促进。

另外,由于社交网络与地点推荐在某些软件下的高度结合,为用户推荐感兴趣的地点也是最近的热点问题。与传统商品、音乐、电影推荐不一样,在地点推荐中,人们的选择不再单纯地由地点的吸引程度来决定,还将由社交网络中用户与朋友的关系和朋友的偏好来决定,此外,用户的日常活动范围也将限制用户的选择。在这样的情境下,如何结合用户的社交网络、地点的地理信息,来给用户推荐准确地进行推荐刚兴趣的地点,是地点推荐系统中最重要的研究问题。另外,挖掘用户社交网络以及地点层级结构中的模式本身就是很重要的数据挖掘任务(社团挖掘、地理区域划分),如果能从用户与地点的交互中,以某种方式以促进社团挖掘以及地理区域划分过程,将更利于用户的地点推荐,形成一个互相促进的正反馈。这使得推荐系统与社团挖掘的性能提升的同时,也令信息的利用效率大大增高。


2. 国内外研究现状:
二十一世纪以来,随着带有GPS、GSM及RFID等移动设备的普及,大量具有时间与空间属性的数据被收集与存储下来,研究如何从这些海量数据中提取出固定模式,催生了大量传统的轨迹挖掘任务。而后,由于社交网络的兴起与多领域数据的整合,轨迹与基于地点的签到数据(check-in data)被关联了大量的语义信息,这也使得轨迹的语义分析与挖掘成为了关注的热点。本节将总结近年来轨迹的语义表征与常见挖掘任务,这其中,结合社交网络的地点推荐系统的研究方法将被单独列举出来。

2.1 轨迹语义表征与挖掘

由不同设备采集到的原始轨迹数据只是一个带着时间戳的序列,其中包含的信息需要进一步筛选和提取。不同的研究工作根据后续的挖掘任务,提出了不同的轨迹特征提取方法以及语义表达模型,此处将常见的轨迹表达模型分类为三种:

(1)基于关键点提取的轨迹表征
将轨迹数据用有限且紧凑的关键点来表达是一种最常见的方式。而关键点有两种定义方式:(a)全局关键点。这种类型的关键点又被成为Points-of-Interest (POIs),通常来说,这些点是提前选定的,比如地图上的学校、餐馆、加油站等标志性点或区域,也可以简单地将地图均匀地划分为小方格得到\cite{giannotti2007trajectory,wei2012constructing,shang2014inferring,wang2014travel,yuan2015discovering,xue2013destination,zheng2015approximate,cho2011friendship}。然而也有很多研究工作致力于从原始轨迹中提取出这样的全局关键点(区域),比如一些工作\cite{ashbrook2002learning,ashbrook2003using}采用基于划分的K-means的思想来对轨迹集中的GPS点进行聚类得到全局关键点,同样地,另外一些工作则使用基于密度方法来聚类,例如DBSCAN,OPTICS,KDE,以及重力模型\cite{li2010mining,zheng2011recommending,jeung2008discovery,chen2011discovering,zheng2008understanding,wang2015regularity}。(b)基于单条轨迹的关键点。这种方式通常是基于后续工作的需求,对轨迹的某种特征进行提取。例如文献\cite{li2008mining,zheng2011learning}提取每一条轨迹的stay points来代表该轨迹,stay points即轨迹进行停留了一段时间的点,能结合停留时间与地点场景,刻画轨迹产生者的行为,如购物、吃饭、睡觉等。而文献\cite{adrienko2011spatial}则提取出重要的驻点以及拐点来作为一条轨迹的关键点。

(2)基于关键线段提取的轨迹表征
类似于关键点表征,关键线段表征也是一种直观的轨迹表示方法。轨迹的关键线段同样可分为两种:(a)全局关键线段。在城市中,车辆的交通轨迹应当被道路所约束,因此将轨迹表示为一系列道路段的拼接是一种直观而合理的方式,这也催生了一些列路网匹配的方法\cite{greenfeld2002matching,chen2003integrated,newson2009hidden},同样,也有从轨迹集合中提取出线段的工作,例如文献\cite{lee2007trajectory}用分段聚类的方法,用最小描述长度(MDL)作为评价因子将轨迹中的代表性线段聚类找出。(b)基于单条轨迹的关键线段。这一类的方法也较为直观,例如文献\cite{zheng2008learning,zheng2008understanding}将每条轨迹划分为行走段和非行走段,而文献\cite{lee2011trajectory,douglas1973algorithms,bellman1961approximation} 则用压缩的思想,将原轨迹段集合表示为最少的特征段,并在这个过程中保持最少的误差。

(3)轨迹高层结构提取与表达
根据后续的挖掘任务,很多工作直接从轨迹数据中提取出高层的数据结构。例如为了轨迹检索,轨迹被投影到了树的结构上\cite{guttman1984r,wang2008flexible,pfoser2000novel},文献\cite{zheng2009mining} 则提出了用户与位置的二分图与树形层次图结构来表征原始轨迹中用户与地点的关系。文献\cite{yuan2015discovering,wang2014travel,liu2016unified}则用用户与地点和时间的三阶张量来对轨迹进行信息抽取。

上面说的三种轨迹表征方法提出目的都是为了后续的轨迹挖掘任务,这里简单列出两种:

(1)轨迹聚类
轨迹聚类的目的是将大量的轨迹划分为几个有限的簇,每一个簇都能直观的体现轨迹的移动模式,因而更好的进行城市规划等任务。最近的一篇综述文献\cite{yuan2017review}给出了一个全面的轨迹聚类方法总结。其中经典的模型有基于时空的轨迹聚类法,如同文献\cite{kisilevich2009spatio}指出的,在轨迹聚类中表示时间信息是很有挑战且必要的,文献\cite{birant2007st}在轨迹聚类中引入了时间周期模式(cyclic time patterns)的概念,而文献\cite{nanni2006time}则引入了时间间隔的概念,在此类文章中,轨迹聚类是在一定的相对时间和绝对时间约束下进行。近年内,轨迹的语义聚类开始盛行,很多方法\cite{yan2013semantic,palma2008clustering,zheng2008understanding,ying2011semantic}将其他领域的语义信息赋予轨迹,使得轨迹聚类与其他挖掘任务关联起来,使得聚类结果有很好的解释性。

(2)社交关系挖掘
分析轨迹数据的其中一个目的就是发掘移动对象的交互,因此达到社团挖掘的目的。其中一些研究者将移动对象的关系刻画为“共现”(co-presence)\cite{crandall2010inferring},还有一些用访问特定点来判断用户的关系紧密程度\cite{wang2011human,gaito2011mobility}。文献\cite{li2008mining}给出了一个结合了用户关系网络与地点网络以及用户与地点的关系交互网络的一般架构,目的在于从用户的行为轨迹中找出其社交模式,从而推断出其潜在朋友,进而补全信息丢失的用户网络,反过来也可以利用用户的关系更好地刻画用户的行为模式\cite{zheng2011recommending}。总结来说其思路是将用户的历史行为表示为T-pattern\cite{giannotti2007trajectory}的形式。然后比较两两用户间的T-patterns序列,将小于一定阈值的用户检索出来作为朋友。此外,还有其他方法\cite{xiao2010finding,zheng2011recommending}用也是这个框架与思路。

存在问题与挑战
现有的轨迹表征方法通常针对某种特定轨迹挖掘任务而设计的,其只考虑了某些特定领域的信息,没有将各种语义信息综合进表征中。且在预处理过程中,每条轨迹都是分开对待的,随着轨迹数据的飞速增长,相似的轨迹与地点将被冗余地处理多次,这种分开处理的策略将占用越来越多的存储空间和处理时间,因此,我们需要提出一种基于全局的轨迹表征。为了使这种全局表征方式更加完备,轨迹的相似性度量以及各种语义都应以某种方式融入到这种表征中来。

2.2 结合社交网络的地点推荐系统

(1)基于纯地点的推荐系统
传统的推荐系统的思想很朴实,给用户推荐商品,并不用局限于社交网络下的地点推荐系统(LBSNs)。然而,由于用户在社交网络中的签到数据的地理信息可以挖掘出用户某些偏好,2011年左右,文献\cite{ye2011exploiting,berjani2011recommendation}开始将传统推荐系统算法应用于推测POI中。此时的推荐仍然是考虑最经典的协同过滤思想,将POIs视为商品,那基于用户的推荐系统\cite{breese1998empirical}和基于商品的推荐系统\cite{sarwar2001item,linden2003amazon}分别考虑相似用户和相似商品进行推荐。之后,随着基于矩阵分解的推荐方法开始流行\cite{mnih2008probabilistic,lee2001algorithms,koren2009matrix},文献\cite{berjani2011recommendation}提出了加约束项的矩阵分解的POI推荐方法,文献\cite{cheng2012fused}基于概率矩阵分解(PMF)与概率因子模型(PFM)的POI推荐系统。

(2)考虑用户网络的地点推荐系统
事实上,基于用户社交关系的推荐系统在LBSN概念前久被广泛使用了,分为基于存储的\cite{golbeck2006generating,massa2007trust,jamali2009trustwalker}与基于模型\cite{jamali2010matrix,ma2008sorec}的两种方法。之后,文献\cite{cheng2012fused,ye2010location}将这些方法推广到了地点推荐系统中,并取得了良好的效果。

(3)考虑地理影响的地点推荐系统
在地点推荐系统中,自然要考虑地理信息对用户决策的影响。1970年的托比第一地理定律指出:“任何东西都有关联的东西,但距离近的东西关联的更多”\cite{tobler1970computer}。这给了地点推荐系统两个指示:(a)用户访问地点应该遵循就近原则。(b)用户对自己喜欢的地点周边的地点更加刚兴趣。同时,大量的文献\cite{gao2012gscorr,liu2013learning,ye2011exploiting,yuan2013time,zhang2015igeorec}也在研究中发现了空间位置聚集的现象,并把这些规律应用于地点推荐系统中,对已有模型进行了改进。



存在问题与挑战
现有方法大多数是利用社交网络、地理信息来对地点进行推荐,然而在推荐的过程中却没有考虑到其他信息的数据是否全面,是否与用户地点交互数据一致的问题。于是一个问题浮现:能否考虑一个地点推荐系统,不仅将社交网络以及地理信息用于推荐,反之还能用推荐信息来补齐社交网络以及地理信息(社团挖掘问题、区域划分问题),这样能使得信息最大效益的利用,而不同的任务也能互相促进。另一方面,无论是社交网络,还是用户地点交互数据都是非常稀疏的,对于check-in数据,其交互元素大多是用户访问特定地点的次数,而大量的空缺值中,如何将用户不感兴趣的负样本从潜在的未访问的元素中区分开来,是一个困扰推荐系统的问题。



拟解决的关键问题:
本文拟解决的问题主要有以下三个方面:
(1) 提出统一而普适的原始轨迹表征模型
 正如

(2) 充分考虑时空轨迹数据集中的各种语义信息,并提出相应的语义压缩、检索、挖掘算法

(3) 在LBSNs中提出地点推荐系统与社团挖掘互促的算法

技术路线
首先本工作将广泛研究近些年来提出的数据流半监督学习模型以及安全的半监督模型,对于数据流半监督模型,重点还需要关注能顺应概念漂移的算法,比如基于实例的数据流算法;而对于安全的半监督模型,重点研究基于自适应权重的模型,避免由于其他类模型的复杂度而难以扩展应用到数据流情况中,尤其注重数据权重的影响因素,学习快读建模方法以及快速优化算法。基于以上研究,分别提出一种基于权重自适应的可靠半监督学习模型,并且将其扩展到数据流环境下,提出一种简单直观的可靠的数据流半监督分类算法。整体的算法架构如下图所示,

\bibliographystyle{IEEEtran}
% argument is your BibTeX string definitions and bibliography database(s)
\bibliography{KaiTi}
%







