% !Mode:: "TeX:UTF-8"


\indent
研究生阶段匆匆拉下帷幕。这三年我仍然在我本科母校电子科技大学度过。比起本科,我成长了许多,也有许多不满意的地方。我在轨迹挖掘上做了一些非常微小,甚至可以说不值一提的工作。我之后又把兴趣点移向了推荐系统,又做了一点水工作。目前又开始进军NLP,希望我能在这个方向上有所建树。总体来说,虽然没有拿得出手的漂亮工作,但这几年也没白费。

首先谢谢我的父亲和母亲,你们是世上最关心我的人。虽然有时候电话里有点唠叨,有点对我指手画脚,但回过头来想,我还是很感激你们的。我希望你们能一直陪着我,和我每天说说琐事。如今,我慢慢走向独立,我可以依靠自己的力量活在这个世界上。我见识的人和事越多,我越能感受到你们对我的好。希望有一天,我能靠自己的力量来保护你们。

其次,我必须对我的科研导师邵俊明老师表达我由衷的敬意和谢意。自从大三我跨入教研室,邵老师就成为我最亲密的老师。邵老师以身作则,让我懂得了什么叫做科研。有时候,当我从邵老师那得到了新的观点时,我会很敬佩邵老师,并默默下决心成为邵老师;有时候,当我不赞同邵老师的观点,我也会和邵老师争论,之后会私下赌气想要证明自己是对的,想要超越邵老师。和邵老师相处的4年多里,邵老师从做实验到写论文,一样一样的传授我,手把手的叫我,持续不断的纠正我,在这个过程中,我渐渐学会了科研。如今,我决心去澳大利亚深造。我还记得我刚见到邵老师时,我迷茫请求邵老师给我工程做;我还记得邵老师不断耐心地和我探讨科研和工程的区别。绝大多数情况下,邵老师是对的。认真想来,可能我在学习效率、思维发散度以及品德上,可能永远超越不了邵老师。我庆幸我有这么一位人生导师。邵老师,谢谢您!

我还要谢谢西竹,与你相识相爱的过程,看似是缘分,从开始以来就没有很大的波折。但这世上并不是所有事都能用缘分解释的,只能说我在恰当的时间遇到了你,让你碰到了最好的我。我想,如果早一些碰到你的话,我是没办法和你走到一起的。究其根本,还是靠努力,并不局限于努力学习,而是努力地去热爱:热爱生活,热爱深蓝的天空,热爱每一本书,热爱跨过每一道坎的感觉,热爱这一切。这种热爱让我觉得人生好短,时间好快。我能在你眼里看到相同的热爱,这份热爱把我们吸到了一起。希望我们能坚守这份热爱,对世界保持好奇,并投入高度热情,做到:学习废寝忘食、吃饭狼吞虎咽,游戏乐不思蜀。

我也要感谢给我上课的每一个老师,你们传授我知识,你们让我看到世界。感谢教研室的同学们,你们让我学会了如何做人做事。我打心底感激我的朋友,与我分享喜悦忧愁的朋友,与我共患难的朋友。篇幅原因我就不写了,好朋友终归会聚到一起的。

最后我要感谢我自己,一直没有放弃的自己。希望今后我关心的人与关心我的人,都能依靠努力获得自己想要的人生!今后,我将努力学会珍惜:珍惜我的健康,珍惜我的家人,珍惜我的爱人,珍惜我的良师益友,珍惜我所拥有的一切。


% \newpage\mbox{}\thispagestyle{empty}\newpage