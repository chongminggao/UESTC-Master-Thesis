% !Mode:: "TeX:UTF-8"

% \addtocounter{page}{-1}
\begin{Eabstract}{Trajectory Compression}{Trajectory Representation}{POI Recommendation}{}{}
Recently, with the pervasive use of mobile devices and the improvement of store and computing capacity, massive amount of trajectory data and check-in data have been generating at a dazzling speed. There are human mobility patterns waiting to be exploited behind those data. Therefore, trajectory mining topics such as storing, compression and representation draw growing attention to provide profound influences in economy, environment, and transportation. In business, the user preference and other knowledge extracted from the sea of trajectory data and check-in data can benefit the location recommendation process. Thus human life quality can be improved, and business with different scales can make a profit out of it.




To handle the massive amount of trajectory data and mine knowledge from it, researchers have proposed a bunch of methods for trajectory compression, trajectory representation and mining algorithms for applications such as location recommendation. However, there are still many challenges to be tackled. The main contributions of this work can be summarized as the following three points.


% this work proposes a trajectory compression and representation method. Different from traditional methods which deal with each trajectory individually, this work globally processes the whole data set. Not only is the process very efficient, but the semantic information that those traditional methods ignore is also captured by integrating the auxiliary urban point of interest (POI) information. By doing this, for one thing, the trajectories located on the spares areas can be regulated and corrected. And for another, the places of interest can be understood well. Furthermore, to overcome the defect that places are represented by coordinate pairs on the map and thus the semantic information cannot be revealed, this work proposes a representation method that project a place to a distributed latent vector. The similarity between two latent vectors stands for the semantic similarity of the two places, and thus the downstream applications can be expedited. At last, this work provides a novel explainable place recommendation method. By locally applying a forwarding stagewise manner matrix factorization on the rating data, the result factors are enriched with meanings and the recommendation results become easy to explain. To conclude, the contributions of this work is as bellows.

\begin{enumerate}
\item ~~~ The existing trajectory compression methods usually focus on simplifying the lines that compose the trajectory, without considering the global information of trajectories. Though some work takes the global semantic information into account, it fails in the area where the semantic information is sparse or missing. We propose an efficient and robust method which can compress trajectories globally. The basic idea is to cluster, with both global trajectory structure information and contextual information served as fixed points if available, all GPS points to common regions at the same time, and transform the trajectory data set into an ROI network. This method provides a new perspective for understanding the semantic the utility of urban areas, and on the other hand, the trajectories located on the spares areas can be regulated and corrected. The experiments show the effectiveness and efficiency of the proposed method.
\item ~~~ The unstructured nature of trajectory data makes measure the similarity between two trajectories difficult. The traditional trajectory representation methods ignore the geometric property and semantic information (network structures, temporal information, and domain knowledge). This work proposes a hierarchical embedding model to embed each region or trajectory as a continuous vector in a semantic vector space. Thereby, the semantic similarity between two regions or trajectories can be measured by computing the Euclidean distance of two vectors directly. In experiments, the proposed representation facilitates tasks of semantic location retrieval and trajectory retrieval.

\item ~~~ To recommend the point of interest to users in the location-based social network is a profitable task. However, the conventional methods only aim at the predicted accuracy, leaving exploring the reason behind the model untackled. In this work, we propose a novel explainable place recommendation model. By locally applying a forwarding stagewise manner matrix factorization on the rating data, we enrich the result factors with meanings and the recommendation results become easy to explain. The experiments illustrate both high accuracy and good explainability at the same time.
\end{enumerate}

Through the effort devoted to the three aspects, we fill the blank of the related research area and pave the road for following researchers.
\end{Eabstract}
% \newpage\mbox{}\thispagestyle{empty}\newpage
% \addtocounter{page}{-1}
