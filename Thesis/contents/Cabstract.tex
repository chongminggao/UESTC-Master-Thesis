% !Mode:: "TeX:UTF-8"

\begin{Cabstract}{轨迹压缩}{轨迹表征}{地点推荐}{}{}
近年来,随着移动设备飞快普及与硬件存储、计算能力的飞快提升,每天都有海量的轨迹数和带地点标签的签到数据以惊人的速度产生。这些数据蕴含着着人们的移动规律以及出行模式,因而高效地对轨迹数据进行存储、压缩、表征以及知识挖掘将对现有经济、环境、交通等领域产生深远影响。在商业方面,探索如何从海量的用户轨迹或签到数据中挖掘出用户喜好信息,进而向用户推荐潜在的感兴趣地点,将使人们的生活得到极大的提升,也能为不同规模的商业经济带来巨大的效益。

针对轨迹数据的特征,本文提出了一种全局的轨迹压缩、表征方法。区别于传统的将轨迹数据逐条压缩处理,本文将对海量的轨迹数据进行全局处理。这种做法不仅非常高效,还能通过结合都市中丰富的已知POI信息来抓住传统方法忽略的全局统计以及语义信息。这样一方面使得轨迹稀疏的区域的轨迹表征得到矫正,另一方面也能借助大量轨迹数据来探索和理解关键地点的语义信息。之后,本文指出,传统的地点表征仍然是直观的基于地图的距离的,没有对潜藏的各种语义进行探索,借助词向量的嵌入表征方法,本文创新性地将地点表征为隐向量,隐向量之间的相似度即可提现地点间的语义相似度,这将为后续应用场景提供非常有效的表征基础。最后,在地点推荐算法层面上,本文指出了传统协同过滤不能够产生有解释性的推荐结果,进而结合了集成学习的的思想,提出一种局部分步矩阵分解的协同过滤算法,弥补了基于矩阵分解的推荐算法在解释性上的空白。总观全文,本文的创新点体现在以下三方面:

第一、本文提出了一种全局的轨迹压缩表征方式,将整个轨迹数据集表征为一个多粒度的地点网络。这个网络可根据应用的需求将已知的地点信息包含进来以增强轨迹的表征压缩效果。

第二、结合多样化的地点挖掘需求,本文提出了一种将地点表征为隐向量的方法,使得地点之间的高层语义相似度可以直接从隐向量之间的相似度中获得。这将大大提高地点检索的效率。

第三、针对传统矩阵分解在地点推荐中缺乏解释性的缺陷,本文提出一种改进型的局部分步矩阵分解。这种方法应用在地点数据集上将让产生的推荐隐因子具有更具体的含义,从而获取用户的信赖程度,也增强了研究者对于算法的理解。

本文通过实验说明了提出方法的可行性,其贡献填补了轨迹数据挖掘与地点推荐一些空白。
\end{Cabstract}

% \newpage\mbox{}\thispagestyle{empty}\newpage
% \addtocounter{page}{-1}
