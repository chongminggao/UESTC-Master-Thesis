% !Mode:: "TeX:UTF-8"

\chapter{全文总结与展望}
\label{chapter:conclusion}
\section{全文总结}
本文我们介绍了目前比较火热的双边聚类问题,提出一种基于同步原理的全新的双边聚类算法\cosync,并在人工数据集以及基因数据集上进行测试,取得了优秀的成果。各章节的主要内容总结如下:

\begin{itemize}
  \item ~~第\ref{chapter:introduction}章为引言部分,从数据挖掘的聚类领域谈起,介绍了双边聚类需求的产生,并正式定义了双边问题以及联合簇。
  \item ~~第\ref{chapter:rw}章的第一部分整理和回顾了国际上相关的知名研究成果。介绍了从两个角度对联合簇具体形式的分类,之后对现有的大部分双边聚类算法进行了分类总结,将它们大致分为基于启发式搜索的方法和非度量式方法,针对每个方法都列举了一些算法的工作原理。接下来第二部分介绍了自然界中同步的概念以及用同步思想作为聚类原理的\sync算法。
  \item ~~第\ref{chapter:main}章开始正式介绍\cosync算法的多个步骤。以同步聚类思想为切入点,我们提出了全新的双边加权交互模型,使得数据集矩阵中的联合簇能够随着时间自动收敛为常数值。在交互模型收敛后,我们提出了一种启发式的同值子矩阵搜索算法来挖掘结果矩阵中的常数值子块。最后,为了能够处理高维数据,避开高维诅咒的困扰,我们引入非负矩阵分解。至此,\cosync算法便能在大规模数据矩阵中进行联合簇挖掘。
  \item ~~第\ref{chapter:experiment}章为实验部分,为了证明\cosync算法的可行性和高效性,我们将双边聚类中有代表性的ITCC,MSSRCC,Spectral Clustering和Plaid算法一起加入实验。实验在人工数据集和基因数据集上展开,最后结果显示\cosync具有极好的性能,超越了其他对比算法。
\end{itemize}


\section{后续工作展望}
本文中我们已经完成了关于双边聚类新算法\cosync的所有工作,之后我们将对全新的工作开展工作,即围绕\textbf{多边聚类}问题进行研究。

考虑这样一个问题,在推荐系统中,我们能得到不同用户在不同时间内,对不同商品的喜好数据,这种数据可以写为一个数据立方体$A$,其中任一元素$A_{ijk}$表示第$i$个用户在$k$个时间段内,对$j$个商品的评分。类比双边聚类,多边聚类即在类似的时间段内,找出相似的用户以及对应的相应的商品。

双边聚类的原理是在数据矩阵中,对每一个元素都用其行列邻居对它进行交互,最终达到同步的状态。那现在拓展这个思想,我们在一个数据立方体甚至更高维数据张量(tensor)\footnote{\url{https://en.wikipedia.org/wiki/Tensor}}中,仍然用这种同步交互的思想,对数据张量中的元素进行动态交互。图\ref{future}给出了在数据矩阵上进行双边聚类以及在数据立方体上进行多边聚类的交互示意图。

\pic[h]{从双边聚类到多边聚类}{}{future}

如图\ref{future}(b)所示,数据立方体中任一元素将被其$x,y,z$三个维度上的邻居影响交互,随着时间迭代最后达到同步的状态。此时的聚类簇结构将表现为数据立方体中包含的常数子块。

关于多边聚类问题,目前国际上研究的成果很少。一方面,真实世界中,不稀疏的数据立方体或者更高维的数据很难获取,另一方面,处理这样的数据困难而效率低下。我们将用同步的思想,对多边聚类问题展开研究,争取在这一领域作出成果,向国际研究前沿进军。路漫漫其修远兮,吾将上下而求索!


