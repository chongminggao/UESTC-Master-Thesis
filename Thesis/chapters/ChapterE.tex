% !Mode:: "TeX:UTF-8"

\chapter{基于局部分步式矩阵分解的可解释性地点推荐系统}
\label{chapter:main3}

推荐系统旨在积极准确地为用户提供有趣的信息或服务。在所有推荐方法中,基于矩阵近似(MA)的协同过滤(CF)方法被证明非常简洁和有效。最初,在只有用户-商品交互评分数据可用的时候,研究人员将用户和商品都看成独立个体并简单地将评分矩阵分解为两个低秩矩阵\cite{paterek2007improving,weightedSVD,mnih2008probabilistic,rennie2005fast,salakhutdinov2008bayesian}。以这种方式,用户的偏好和商品的属性便可以体现在这低秩的潜在向量中。当通过取用户向量和商品向量的内积时候,缺失的评分值便可以被预测出来。


\section{}


