% !Mode:: "TeX:UTF-8"

\chapter{绪论}
\label{chapter:introduction}
\section{引言-从数据挖掘谈起}
% 计算电磁学方法\citeup{wang1999sanwei,
% liuxf2006,
% zhu1973wulixue,
% chen2001hao,
% gu2012lao,
% feng997he}
% 从时、频域角度划分可以分为频域方法与时域方法两大类。
% 频域方法的研究开展较早,目前应用广泛的包括:矩量法(MOM)\citeup{xiao2012yi,zhong1994zhong}及其快速算
% 法多层快速多极子(MLFMA)\citeup{clerc2010discrete}方法、有限元(FEM)\citeup{wang1999sanwei,zhu1973wulixue}方法、自适应积分(AIM)
% \citeup{gu2012lao}方法等,这些方法是目前计算电磁学商用软件
% \footnote{脚注序号“①,……,⑩”的字体是“正文”,不是“上标”,序号与脚注内容文字之间空1个半角字符,脚注的段落格式为:单倍行距,段前空0磅,段后空0磅,悬挂缩进1.5字符;中文用宋体,字号为小五号,英文和数字用Times New Roman字体,字号为9磅;中英文混排时,所有标点符号(例如逗号“,”、括号“()”等)一律使用中文输入状态下的标点符号,但小数点采用英文状态下的样式“.”。}
% (例如:FEKO、Ansys 等)的
% 核心算法。由文献\citeup{feng997he,clerc2010discrete,xiao2012yi}可知……


从轮子的发明,到登月火箭的实现,上千年来的科技的进展给人类的交通方式带来的非常大的改变。越来越多的通信技术开始关移动物体的轨迹。现如今,我们在每一天我们使用计算机、手机时候,都有大量数据产生,接着被以各种形式记录、保留下来。这其中,大量数据是带有地点标签的,这些数据随着时间扩展,便可得到人们日常出行的轨迹以及这个过程中的各种信息。这些信息记录了我们在什么时间,什么地点做了什么东西,利用好这些移动数据,将给未来人们的生活带来革命性的改变。

那么,如何利用好记录了移动模式的轨迹数据进行挖掘和探索呢,本文在将首先介绍轨迹数据挖掘中的一些任务,明确目前轨迹数据中存在的问题,再介绍目前国际上存在的主流方法与技术以及这些技术存在的问题。针对这些问题,本文将提出创新的的方法来对目前的算法进行改进以填补这部分应用的空白。接下来,轨迹的产生的以及记录形式将被正式的介绍。

\subsection{轨迹数据的产生于记录}
由于记录设备的不同,轨迹的存在形式可以由多种。Spinsanti等人\cite{spinsanti2013mobility}将轨迹数据的形式区分为基于GPS(global positioning system),GSM(global system for mobile communications)和基于社交网络的轨迹这三种。Pelekis和Theodoris\cite{pelekis2014mobility}又追加了两种轨迹数据,分别为基于RFID(radio frequency identification)的和基于Wi-Fi数据的。这其中,基于GPS系统的轨迹数据由一系列带时间戳的二维地理坐标构成。基于GSM的轨迹数据由一系列带时间戳的物体经过的细胞标号组成。而基于RFID的轨迹数据包含物体经过的一系列RFID接收器的标号组成,基于Wi-Fi数据的轨迹也一样由物体连接通讯过的一系列Wi-Fi基站标号组成。不同形式的轨迹数据的精度是不同的,它们在不同的应用场景下有着不同的应用。

通常,一条轨迹总是可以被表示为:




在这样多领域的需求下,数据挖掘(Data Mining)这门交叉学科应运而生。通常来说,数据挖掘是数据库知识发现(Knowledge-Discovery in Databases)中的一个步骤,其目的是在大量的数据中自动搜索隐藏于其中的特殊信息,从而为之后的分析决策提供理论依据。下面将简要介绍下数据挖掘的主要步骤:
\vspace{4mm}
\pic[h]{数据挖掘主要步骤图(来源:Synchronization Inspired Data Mining\citeup{shao2011synchronization})}{}{overview}
\begin{itemize}
    \item \textbf{~~数据采集} 所有工作开始之前,首先需要采集数据,包括确定数据种类、范围等,然后对数据进行初步选择,挑选出合适的数据。
    \item \textbf{~~数据预处理} 该过程包括对原始数据的处理,包括数据整合、去除噪声等。
    \item \textbf{~~数据转化} 对数据进行完预处理后,需要决定数据合适表示,例如特征选筛等。
    \item \textbf{~~数据挖掘} 这个过程中,人们采用各种方法,例如聚类、分类、关联规则分析等方法来发掘数据中的有用的信息。
    \item \textbf{~~结果评估与可视化} 最后,需要对得到的结果进行解释与评估,并可视化为易于人理解的形式,在这之后有可能需要重新进行挖掘。
\end{itemize}

\vspace{2mm}
这其中,\textbf{数据挖掘}是从数据中学习知识的最关键的步骤,因此很多时候,数据挖掘泛指从数据中学习知识的过程。数据挖掘的大量算法可以按照目的分为以下四类:

\begin{itemize}
    \item \textbf{~~分类(Classification)}分类算法的目的是为特定变量确定类别或者标签,比如根据近年来我国的经济发展情况来确定房价是涨还是跌。一般来说,分类首先用历史数据作为训练集,学习出目标函数,然后用学到的目标函数来预测新来的未知数据点的类别。常见的分类算法有kNN\citeup{peterson2009k},决策树\citeup{quinlan1986induction},支持向量机\citeup{cortes1995support}等。
    \item \textbf{~~聚类(Clustering)}聚类算法的目的是将数据分为许多类,使得相似的数据分在同一类中,不相似的数据分布在不同的类中,比如菜农可以根据一批辣椒的形状、辛辣程度将其聚拢成不同类别销售。常见的聚类方法有k-means\citeup{hartigan1979algorithm}, spectral clustering\citeup{ng2002spectral}和DBSCAN\citeup{ester1996density}等方法。
    \item \textbf{~~关联规则分析(Analysis of Association Rule)}关联规则分析的目的是从数据中发现经常出现的模式,一个经典的例子是人们从超市的大量销售记录中发现买尿布的人也常常买啤酒。经典的关联规则分析方法有:Apriori\citeup{agrawal1994fast}和FP-growth\citeup{han2000mining}等。
    \item \textbf{~~奇异点检测(Animation Detection)}奇异点检测的目的是发现数据集中存在的奇异点,即与大多数点不相似的少数数据点,比如邮件代理公司会根据正常邮件与垃圾邮件的特征对比,来为用户标记垃圾邮件。通常来说大多数聚类算法都可以作为奇异点检测算法。
\end{itemize}

\vspace{2mm}
相对于数据挖掘的其他算法,聚类的知识目前还不够系统化。一个重要原因是聚类不存在客观标准:给定数据集,总能从某个角度找到以往算法未覆盖的某种标准从而设计出新算法\citeup{estivill2002so}。但聚类技术本身在现实任务中非常重要,近些年关于聚类的新算法在数据挖掘、机器学习、人工智能的顶级会议乃至《自然》和《科学》上都频出不穷。本文也将提出一种全新的基础聚类算法,在此之前,先引入由特殊需求引入的新型聚类技术:双边聚类技术。



\section{双边聚类技术}
聚类技术有很多变种,其中双边聚类(Co-Clustering,或Bi-Clustering,Two-mode clustering)就是一种,其致力于突破传统聚类的限制,在两个空间中同时进行聚类,从而创造更好的应用价值。现在首先来谈谈双边聚类是什么,随之引入一个正式的双边聚类的问题描述。

\subsection{“同时聚类”需求的产生}
在传统的聚类中,对于一个数据集,总是给定一个特征空间,对数据集进行聚类。比如在传统的“用户-商品”推荐系统(Recommondaton System)中,想对用户进行聚类,那就要将各种商品作为特征集,通过不同用户喜欢的商品集合的异同来判断用户之间的相似性,从而最终达到对用户聚类的目的。同样的,如果想对商品集合进行聚类,那反之得将商品集合作为数据集,用户作为特征集,对商品进行聚类。至于聚类的目的,对同一个用户簇可以推荐相同的商品,而同一个商品簇可以归类到一起进行管理,这是后话了。

同样的,对于文本挖掘(Text Mining),大量的词汇和文档也可以组成两个空间,将其中一个空间作为特征集,对另外一个进行聚类,此类例子还有很多。那么,有的时候,我们不禁发问:能否同时对两个空间进行聚类?比如在推荐系统中同时对用户集合和商品集合进行聚类,于是在得到相似的用户的群组的同时,得到该群组喜欢的商品集合!同样地,在文本挖掘中我们是否可以在得到相似的文本集的同时,得到该文本集包含的词汇集?

要是这种两个空间“同时聚类”的问题能够解决,那么在大量应用中将得到极好的结果和可解释性,甚至颠覆传统聚类方法的价值,从而为科研和生产提供全新的方向。事实上,现在已经有大量的双边聚类算法诞生并且投入应用,先来看看最初的双边聚类算法是怎么出现的。

\subsection{双边聚类问题描述}
双边聚类问题\footnote{\url{https://en.wikipedia.org/wiki/Biclustering}}诞生于生物信息学中的基因表达问题(Gene Expression Profiling)上,简单来说,近年来生物信息检测技术的进步为科学家们提供了大量的基因表达数据,即大量的基因在不同样本(不同个体、不同组织或者不同环境)的表达的程度,可以用一个“基因-样本”的矩阵$A$来表示,其中的元素$a_{ij}$表示编号为$i$的基因在样本$j$下的表达程度,数值越大表达的程度越大。

依据传统的聚类技术,我们可以将基因进行聚类,体现在矩阵中形成横向的簇,如图\ref{gene-condition}(a)所示;也可以将样本进行聚类,体现在矩阵中形成纵向的簇,如图\ref{gene-condition}(b)所示。

\pic[!htb]{数据挖掘主要步骤图}{width=150mm}{gene-condition}

然而在生物信息学中,我们最需要的信息是哪些特定的基因在哪些条件下会一起比较大程度的表达,从而在后面的转录、翻译程序中形成我们关心的RNA、蛋白质。此时我们需要的聚类结果如图\ref{gene-condition}(c)。

根据以上的基因表达问题,我们可以正式定义双边聚类这一问题,首先我们定义联合簇:
\begin{dingyi}[联合簇(Bicluster)]
\label{dingyi:bicluster}
在任一$M\times N$的关联矩阵$A$中,行的集合为$V$,列的集合为$U$。任取$V$的一个子集$I$,$U$的一个子集$J$,则$I$和$U$可以组合成矩阵$B$,使之成为$A$的一个子矩阵。这个矩阵$B$即为一个联合簇。
\end{dingyi}

联合簇具体的形式和产生机制因算法不同而不同,在后一章节的第\ref{sec:bicluster}节将会详述。而双边聚类即找到矩阵$A$中所有的联合簇(子矩阵)的过程。


这一类问题,早在2002年,就被Tanay及其同事\citeup{tanay2002discovering}证明为NP难问题,故大部分的算法都采取启发式的搜索的手段来解决此问题。关于联合簇的具体形式和条件,在不同的方法中,有不同的定义。更多的信息可以参看这篇2005年的综述\citeup{tanay2005biclustering}。关于相关算法细节介绍将在后一章节的第\ref{sec:algorithms}节中介绍。

\section{本文主要贡献与创新点}
本文的核心工作是提出了一种算法\CoSync{},以全新的双边聚类思维方式:动态的方式,使矩阵中元素进行自发地变化达到聚合,从而得到联合簇。这种方法区别于目前所有的双边聚类算法,细节将在第\ref{chapter:main}章中进行介绍。其贡献与创新点如下:
\begin{enumerate}
  \item \textbf{~~全新的视角}:\CoSync{}找寻联合簇的方式非常规的启发式搜索,而是利用全新的视角:动态模拟(dynamic simulation)。该方法建立在两个聚类空间的加权交互上,使得隐藏在矩阵中的、能体现相关生物学模式的联合簇能够直观地自动浮现出来。
  \item \textbf{~~抓住内在自然结构}:\CoSync{}方法对数据集的结构和联合簇的形状没有假设限制,其工作原理使得找寻到的联合簇是严格的数据本身内在的结构体现。
  \item \textbf{~~克服了高维诅咒}:对于高维矩阵,\CoSync{}结合了非负矩阵分解(NMF)的相关技术,将其分解为两个低维矩阵,同时保留了主要信息。此举使得高维矩阵的计算原本极高的时间复杂度降低了不止一个量级,从而让\CoSync{}可以对大规模真实数据集问题求解。
\end{enumerate}

\section{本文的结构组织与章节安排}
本章从数据挖掘学科讲起,聚焦到具体的子分支:聚类问题上,由特殊的“同时聚类”的需求引入了双边聚类问题,并做了正式的定义。接下来章节将安排如下:

\vspace{2mm}
\begin{itemize}
  \item ~~第\ref{chapter:rw}章为相关工作,其将对目前所有的双边聚类算法做一个简要的综述,根据其定义的联合簇结构和算法工作原理将其分成几个分支分别介绍,并指出它们算法的内在缺陷。同时还将介绍本文动态思想的来源:\Sync{}算法的思想和工作原理。
  \vspace{1mm}
  \item ~~第\ref{chapter:main}章为本文的主体方法,分布提出了\CoSync{}算法中包含的模型:加权双边交互模型、最大同值子矩阵搜索算法和非负矩阵分解算法。
  \vspace{1mm}
  \item ~~第\ref{chapter:experiment}章为本文的实验验证部分,首先指定算法工作的人工数据集和真实数据集,之后定义衡量算法好坏的评价指标,接着用一系列的实验来说明\CoSync{}\\算法的效果,并对每个实验结果进行说明。
  \vspace{1mm}
  \item ~~第\ref{chapter:conclusion}章为总结和展望部分,总结了这篇文章的主要工作,给出客观的评价。最后给出了本工作没有涉及的部分和之后可以继续深入做下去的一些工作。
\end{itemize}

\newpage\mbox{}\thispagestyle{empty}\newpage
