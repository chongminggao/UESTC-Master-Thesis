% !Mode:: "TeX:UTF-8"

\chapter{绪论}
\label{chapter:introduction}
% 计算电磁学方法\citeup{wang1999sanwei,
% liuxf2006,
% zhu1973wulixue,
% chen2001hao,
% gu2012lao,
% feng997he}
% 从时、频域角度划分可以分为频域方法与时域方法两大类。
% 频域方法的研究开展较早,目前应用广泛的包括:矩量法(MOM)\citeup{xiao2012yi,zhong1994zhong}及其快速算
% 法多层快速多极子(MLFMA)\citeup{clerc2010discrete}方法、有限元(FEM)\citeup{wang1999sanwei,zhu1973wulixue}方法、自适应积分(AIM)
% \citeup{gu2012lao}方法等,这些方法是目前计算电磁学商用软件
% \footnote{脚注序号“①,……,⑩”的字体是“正文”,不是“上标”,序号与脚注内容文字之间空1个半角字符,脚注的段落格式为:单倍行距,段前空0磅,段后空0磅,悬挂缩进1.5字符;中文用宋体,字号为小五号,英文和数字用Times New Roman字体,字号为9磅;中英文混排时,所有标点符号(例如逗号“,”、括号“()”等)一律使用中文输入状态下的标点符号,但小数点采用英文状态下的样式“.”。}
% (例如:FEKO、Ansys 等)的
% 核心算法。由文献\citeup{feng997he,clerc2010discrete,xiao2012yi}可知……


从轮子的发明,到登月火箭的实现,上千年来的科技的进展给人类的交通方式带来的非常大的改变。越来越多的通信技术开始关移动物体的轨迹。现如今,我们在每一天我们使用计算机、手机时候,都有大量数据产生,接着被以各种形式记录、保留下来。这其中,大量数据是带有地点标签的,这些数据随着时间扩展,便可得到人们日常出行的轨迹以及这个过程中的各种信息。这些信息记录了我们在什么时间,什么地点做了什么东西,利用好这些移动数据,将给未来人们的生活带来巨大改变。

那么,如何利用好记录了移动模式的轨迹数据进行挖掘和探索呢,本文在将首先介绍轨迹数据挖掘中的一些任务,明确目前轨迹数据中存在的问题,再介绍目前国际上存在的主流方法与技术以及这些技术存在的问题。针对这些问题,本文将提出创新的的方法来对目前的算法进行改进以填补这部分应用的空白。接下来,轨迹的产生的以及记录形式将被正式的介绍。



\section{轨迹数据挖掘}
% \subsection{数据挖掘流程}
% 数据挖掘(Data Mining)是一门综合性的学科。通常来说,数据挖掘的目的是在大量的数据中自动搜索隐藏于其中的特殊信息,从而为之后的分析决策提供理论依据。数据挖掘的主要步骤为:
% \vspace{4mm}
% \pic[h]{数据挖掘主要步骤图(来源:Synchronization Inspired Data Mining\citeup{shao2011synchronization})}{}{overview}
% \begin{itemize}
%     \item \textbf{~~数据采集} 所有工作开始之前,首先需要采集数据,包括确定数据种类、范围等,然后对数据进行初步选择,挑选出合适的数据。
%     \item \textbf{~~数据预处理} 该过程包括对原始数据的处理,包括数据整合、去除噪声等。
%     \item \textbf{~~数据转化} 对数据进行完预处理后,需要决定数据合适表示,例如特征选筛等。
%     \item \textbf{~~数据挖掘} 这个过程中,人们采用各种方法,例如聚类、分类、关联规则分析等方法来发掘数据中的有用的信息。
%     \item \textbf{~~结果评估与可视化} 最后,需要对得到的结果进行解释与评估,并可视化为易于人理解的形式,在这之后有可能需要重新进行挖掘。
% \end{itemize}

% \vspace{2mm}
% 这其中,\textbf{数据挖掘}是从数据中学习知识的最关键的步骤,因此很多时候,数据挖掘泛指从数据中学习知识的过程。关于数据挖掘的更多信息,读者可查阅经典的综述\citeup{fayyad1996data}和\citeup{han2011data}。

\subsection{轨迹数据的形式}
由于记录设备的不同,轨迹的存在形式可以由多种。Spinsanti等人\citeup{spinsanti2013mobility}将轨迹数据的形式区分为基于GPS(global positioning system),GSM(global system for mobile communications)和基于社交网络的轨迹这三种。Pelekis和Theodoris\citeup{pelekis2014mobility}又追加了两种轨迹数据,分别为基于RFID(radio frequency identification)的和基于Wi-Fi数据的。这其中,基于GPS系统的轨迹数据由一系列带时间戳的二维地理坐标构成。基于GSM的轨迹数据由一系列带时间戳的物体经过的细胞标号组成。而基于RFID的轨迹数据包含物体经过的一系列RFID接收器的标号组成,基于Wi-Fi数据的轨迹也一样由物体连接通讯过的一系列Wi-Fi基站标号组成。不同形式的轨迹数据的精度是不同的,它们在不同的应用场景下有着不同的应用。

通常,一条轨迹总是可以被表示为一下形式:
\begin{equation}
\label{eq:traj}
T = \{\left<s_1,s_2,\cdots,s_n\right>|s_i=(P_i,t_i)\},
\end{equation}
其中\gls{P}代表一个位置或者一个区域的标号,在基于GPS的系统中,$P_i = (x_i,y_i)$ 表示一个GPS经纬度坐标,也是地图上的一个采样点。而$t_i$是$P_i$的采样时间。$n$代表了轨迹$T$的采样数目。例如,图\ref{Hurricane_raw}可视化了大西洋在1851年-2018年间,每年飓风的移动轨迹\footnote{数据来源:\url{https://www.nhc.noaa.gov/data/}}。图中每一条线即为一个轨迹,而轨迹上每一个圆圈代表一个采样点,在此处即为飓风在某一个月内的定位。

\pic[!htb]{大西洋飓风1851年-2018年轨迹可视化图}{width=90mm}{Hurricane_raw}



\subsection{轨迹数据挖掘}
除了这些经典的数据挖掘以外,也有研究者提出了在地理数据上进行数据挖掘的的综述\citeup{mennis2009spatial,han2009geographic,miller2008geographic}。区别于传统数据挖掘,它们主要考虑了方法在空间上的依赖性以及空间属性与非空间属性的结合。值得一提的是,这些工作都没有很好的考虑轨迹的时间特征。

根据Zheng等人\citeup{zheng2015trajectory}的划分,在轨迹数据上进行的数据挖掘步骤一般有四个任务:


\begin{enumerate}
    \item \textbf{~~消除轨迹不确定性:}现实生活中物体的运动轨迹是连续的,但是由于内存、采样率等原因,我们得到的轨迹数据往往是离散的,因此在两个离散时间点之间的轨迹的运动状态就存在一定的不确定性。针对这个问题,一方面,为了提高轨迹数据挖掘的质量,一些工作致力于降低轨迹的不确定性\citeup{cheng2004querying,emrich2012querying},另一方面,为了保护用户的隐私,一些工作致力于扩大轨迹的不确定性\citeup{emrich2012querying,xue2013destination}。
    \item \textbf{~~轨迹模式挖掘:}我们可以从大量的轨迹中挖掘轨迹的模式(单个用户的轨迹模式或者是群体的轨迹模式)。与常规的数据挖掘任务类似,轨迹模式挖掘任务包括运动模式挖掘\citeup{gudmundsson2004efficient}、轨迹聚类模式挖掘\citeup{kharrat2008clustering}、轨迹运动周期挖掘\citeup{cao2007discovery}等等。
    \item \textbf{~~轨迹分类:}针对大量的轨迹数据,我们可以使用传统的监督学习方法对轨迹进行分类。按照不同的分类目的可以将轨迹分为不同的类别\citeup{sohn2006mobility},利用按照交通方式将轨迹分为“步行”、“开车”、“火车”等类别\citeup{zheng2008understanding},或者按照轨迹出行目的将轨迹分为“工作路线”、“旅游路线”等。
    \item \textbf{~~轨迹异常检测:}轨迹异常检测就是寻找与其他轨迹显著不同的轨迹\citeup{lee2008trajectory}。例如交通事故发生的路段产生的轨迹就可以被看作异常轨迹,对同一个用户来说,其某一天的轨迹因为与大部分时间的轨迹差异较大也有可能会被判定为异常轨迹。
\end{enumerate}

现如今,大多数工作把重心放在轨迹数据挖掘上,却忽略了底层轨迹的表征问题。因为不同于其他领域的数据,轨迹数据是非结构数据,其长度不确定,其采样率不确定,就连轨迹的相似度计算都是一个问题。于是轨迹的存储和表征将是一个问题。为了解决这些问题,本文将在下一章节中介绍轨迹数据常用的表征方式。接下来,另一个高层次的应用:基于地理位置的可解释性推荐系统将在这里引入介绍。

\section{地点推荐系统}
推荐系统的目的为用户推荐其潜在感兴趣的商品、信息或者服务,已经得到了大量学术界和工业界的关注。如今的生活里,拿起手机,打开音乐App,当日刚兴趣的歌单已经生成;打开购物App,令人眼花杂乱的首页展现的结果已经经过了算法的高度定制,从而很大概率上是用户会喜爱继而购买的物品;打开新闻App,今日新闻列表里的主题都是接着昨天用户感兴趣的话题而展开的。对于用户,找到自己想要的信息越来越难;另一方面对于商家来说,让自己的产品脱颖而出也变得困难。推荐系统的提出解决了这一困难。基本上,推荐系统已经成功地融入了各个领域:网络购物(Amazon、淘宝、天猫)、信息检索(Google、百度、Bing)、社交网络(Facebook、Twitter、微信、QQ)、特定地点服务(大众点评、高德地图、百度地图、Yelp、Foursquare)、新闻推送(Google News、今日头条、腾讯新闻)等等。

\subsection{常见推荐方法分类}
传统的推荐方法主要包括基于用户的推荐方法、协同过滤与混合的推荐方法。由于对理解有帮助,以下对三种方法的思想做简要介绍:
\begin{enumerate}
    \item \textbf{~~基于内容的推荐:}根据用户已经进行过交互(评分或点击)的商品,找到在内容上相似的其他商品作为推荐结果。用户与商品的交互分为两种情况:显式反馈,例如评分、喜欢或者不喜欢;和隐式反馈,例如观看、搜索、点击、购买等行为。算法通过这些反馈结果预测用户对商品的偏好情况,并通过此偏好判断出用户喜欢的商品属性特征,并根据这个计算候选商品在这些特征上的匹配程度,根据匹配程度对待预商品进行排序后给用户推荐潜在感兴趣的商品。这一方法不要求密集的交互评分信息,但要求商品的属性信息不能太稀疏。此外,商品的冷启动问题也能在这个方法下得到解决:新来商品只需要得知信息就能进行推荐。这种方法的缺陷在于提取的特征将很大程度上决定算法的性能。

    \item \textbf{~~协同过滤:}这个方法类似于现实中口口相传的原理。协同过滤利用相似用户之间具有相似兴趣偏好的方法的这一特点来发现用户对商品的偏好。主要包括基于记忆和基于模型这两种类型,基于记忆式的算法首先通过用户的历史评分差异来对用户(或商品)之间的相似度进行计算,然后再根据用户对商品的历史交互和这一相似度来预测用户对新的商品的评分值。基于模型的方法在形式上更加简洁,其对用户的偏好进行建模,再匹配符合用户偏好的商品。协同过滤由于只需要用户对商品的交互数据,形式简单效果突出,成为了目前应用最为广泛应用的推荐算法。Koren\citeup{koren2008factorization}在2008年提出了基于矩阵分解的潜因子模型(Latent Factor Model)后,这种方式成了工业界最通用的模型。但这种方式也由于用户商品的评分数据相对于商品的总数量太少,遭遇数据稀疏的问题。此外,此类方法存在冷启动问题。

    \item \textbf{~~混合推荐方法:}由于现实中的推荐模型都有各自的不足,为了达到很好的推荐效果,人们常常结合多种推荐算法而形成一个全面的推荐系统。根据结合的顺序,混合推荐方法可以分为三种类型:后结合、中结合和前结合。其中,基于后结合的混合推荐系统的主要策略是用投票或者线性组合的方式来将两种或以上的推荐算法组装起来;而基于中结合的推荐算法的思想是将一种推荐算法有机地融入进另外一个推荐算法中,来取长补短,举个例子:将基于内容的推荐算法融入进协同过滤的算法中,就可以缓解数据稀疏问题;而基于前结合的策略本质上是特征层的结合,直接将各路特征作为输入汇总而进行推荐。例如,在这个同一模型里,将用户所有属性以及行为数据作为输入来产生推荐结果。
\end{enumerate}

以上方法是传统也是最经典的推荐方法,而现如今,随着需求的增多,研究者们纷纷考虑结合更多除了交互信息以外的额外信息来辅助推荐,这样一方面将提升推荐的准确性,另一方面在数据融合后,应用的可能性也大大增多了。通常来说,传统的推荐算法只利用了用户—商品的交互或者评分矩阵,而在地点推荐领域,其特殊性在于有大量的时空信息可用以辅助推荐。接下来,本文将介绍地点推荐系统的基本知识。


\subsection{地点推荐系统}
由于目前科技的发展,人们的各种活动数据、社交数据以及出行数据都可联系起来,形成一个网络,被称为LBSN(Location-based Social Network)。地点推荐,也被称为POI(Point of Interest)推荐,即为用户提供下一个落脚的地点,或者探索可能感兴趣的地方,是LBSN中非常重要的一环。
虽然传统的推荐方法已经很成熟,但其无法完全移植到地点推荐的场景汇总。追根揭底,地点推荐场景与传统推荐场景的不同因素体现在三个不同方面:

\begin{enumerate}
    \item \textbf{~~地理因素:}如同Tobler第一定律\citeup{tobler1970computer}中描述的:“所有事物都与其他事物有有着关联,但通常与更近的事物关联更紧密”。在LBSN中,Tobler第一定律意味着用户会更倾向于访问其附近的地点而非遥远的地点,而且用户可能会访问其喜欢的地点附近的地点。地理因素是地点推荐中最重要的一个因素,其体现了现实中用户的心理特点与行为模式。

    \item \textbf{~~隐式反馈与数据稀疏问题:}在传统的推荐系统里,用户通常会显示地对商品(如书籍、电影等)提交一个分数评价,这个分值通常有一个范围限制,如[1,5],体现出了用户对各种商品的喜爱程度,更高的评分意味着更强烈的喜爱。但是,在签到数据集中,用户与地点的交互是没有评分的,而是用一个频率计数来代替。这个频率是没有上限限制的,比如用户有可能访问一个地方上千次,而其他地方仅仅访问几次。另外,数据的稀疏性也是大问题,比如传统推荐数据集Netflix的稀疏程度为$99\%$左右,而Gowalla数据集有数值的地方仅占整个数据集的$2.08 \times 10 ^ { - 4 }$左右。

    \item \textbf{~~社交影响:}在LBSN中,一个很容易想到的假设是,朋友之间的偏好会互相收到影响,很多工作会利用这个假设来用用户朋友们的行为轨迹来影响该用户\citeup{ma2008sorec,jamali2010matrix}。一些研究者的工作中表明,社交因素将会影响推荐系统。然而这种印象也不是绝对的,Ye等人的工作表明$96\%$的用户共享不到$10\%$的兴趣,绝大多数用户是没有共同兴趣点的。故社交信息在地点推荐中并不像地理位置那样占重要因素。
\end{enumerate}

在下一章节,本文将介绍更多关于可推荐系统推荐算法的细节。现在,我们将总结本文的贡献:

\section{本文主要贡献与创新点}
本文的核心工作围绕着轨迹数据与用户签到数据展开,其贡献与创新点如下:
\begin{itemize}
  \item \textbf{~~全局轨迹压缩方法:}随着轨迹数据的数量级增长,轨迹的压缩成为了存储、处理的必要预处理步骤。通常的轨迹压缩算法通常将轨迹进行单条压缩,这样的缺陷是效率低,会丢失轨迹数据整体的统计信息。本文利用了\Sync算法,将轨迹数据集进行整体压缩,从下到上压缩得到不同粒度的结果。根据需求不同,不同粒度的结果蕴含着不同的语义。这样压缩还有一个好处,城市中已知的重要节点可以直接插入到压缩结果中,成为压缩后轨迹网络的一个节点。根据轨迹随时间增长的这一特点,本文还将这一轨迹压缩方法扩展到了数据流上。
  \item \textbf{~~区域与轨迹分布式表征:}至今,轨迹的相似度都是一个难题,而如何表示轨迹的语义级别的相似度更是一个问题。受到自然语言处理中的词向量嵌入的启发,本文将轨迹作为上下文信息将压缩得到的轨迹网络上的区域上的节点嵌入为一个隐向量。在这个嵌入的过程中,根据实际场景需求,各种语义相似度都可以考虑进来。让后续的地点推荐等应用更为便利。
  \item \textbf{~~提出一种可解释的地点推荐算法:}传统的地点推荐多是基于协同过滤中的矩阵分解模型,虽然这种方式准确率很高,但其分解出的隐因子是没有含义的。从某种意义上来说,这种模型是一种黑箱模型。在本文中,为了克服这个问题,我提出了一种可解释的矩阵分解算法。具体的做法是将原始的矩阵分解修改为了局部分步分解,让每次分解出的隐因子都有对应的含义。且由于每次分解的秩的降低,矩阵分解的效率将大大增加。
\end{itemize}

\section{本文的结构组织与章节安排}
本章从轨迹数据挖掘这一应用开始介绍,介绍了轨迹数据的形式,以及轨迹数据集上的一些挖掘任务。进一步地,关注点从原始轨迹到用户签到数据后,地点推荐任务和常见思想也被做了介绍。接下来的几章的安排如下:

\vspace{2mm}
\begin{itemize}
  \item ~~第\ref{chapter:rw}章为相关工作,将分别对国际上主流的轨迹数据集上的压缩、表征算法,以及签到数据上的的地点推荐算法进行介绍。并对现有问题进行概括,为后面正文的动机做出铺垫。
  \vspace{1mm}
  \item ~~第\ref{chapter:main1}章为本文的第一个主体方法,提出了全局轨迹压缩算法,并将其扩展到数据流上,最后用实验来证明我们全局轨迹压缩算法的可行性。
  \vspace{1mm}
  \item ~~第\ref{chapter:main2}章紧接着上一张接着对轨迹的表征进行探索,在介绍其丰富的应用潜力后,本文将区域以及轨迹表示成为了带语义的隐向量。本文用轨迹上的检索任务来证明了提出算法的可行性。
  \vspace{1mm}
  \item ~~第\ref{chapter:main3}章的探索领域从原始轨迹上升到了特定地点的推荐。在这章中,一种具有解释性的算法被提了出来,并结合LBSN中丰富的社交信息、地理信息,综合成了一种全面的,能让用户信任的算法。在这一章中,我将我的算法与国际主流的几种算法进行对比,证明了提出了算法的可行有效性。
  \vspace{1mm}
  \item ~~第\ref{chapter:conclusion}章为总结和展望部分,总结了这篇文章的主要工作,给出客观的评价。最后给出了本工作没有涉及的部分和之后可以继续深入做下去的一些工作。
\end{itemize}

\newpage\mbox{}\thispagestyle{empty}\newpage
