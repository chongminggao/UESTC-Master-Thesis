% !Mode:: "TeX:UTF-8"

\chapter{总结和展望}
\label{chapter:conclusion}

\section{本文总结}
随着科技的发展,人们物质水准的提高,越来越多的设备连上了互联网,每天记录着人们的移动轨迹。充分地压缩保存、并挖掘这些应用将对很多重要研究问题产生帮助。很多研究者都在轨迹压缩、轨迹表征、基于社交网络的地点推荐中做出过贡献。本文针对传统轨迹压缩、表征的缺陷,提出了一种将轨迹进行全局压缩的方法,通过这种表征方式,轨迹的全局语义信息与额外提供的语义节点将被压缩至一个多层次的区域网络中,这种算法最大的优势在于可以在语义信息稀疏或缺失的数据集中发挥作用。借鉴自然语言处理中的嵌入模型,本文进一步将不定长、杂乱无章的轨迹数据表征成为定长的向量。通过将需要的语义信息嵌入到区域与轨迹的向量中,轨迹之间特定的相似度可以直接由其表征向量的欧氏距离中得出,这使得相似轨迹的检索拜托手工构造查询项,且更加高效。

之后,本文进一步提出了目前基于协同过滤算法的推荐系统的缺陷:没有解释性,整个类似于一个黑箱模型。针对这个问题,本文提出了一种分步局部矩阵分解,来发掘矩阵分解中隐因子的含义。这种模型不仅有着传统矩阵分解模型的高准确性,还有着的可解释性,使得分步分解出的模型中的隐因子都包含了特定含义。


\section{未来工作展望}
在揭示了基于协同过滤的推荐系统可以包含可解释性后,后面的内容可以围绕着推荐模型的可解释性与交互性展开。具体地,目前的推荐系统都是从用户购买历史出发,对用户的偏好进行建模后进行推荐。这样可能存在的问题在于不能实时抓住用户的兴趣爱好。为了改善这一点,我们可以考虑用户在网上留下的大量文字数据、甚至可以用智能对话的方式与用户进行交互式的问答,从而让推荐算法新上一个台阶,也让人工智能更加理解人类需求。